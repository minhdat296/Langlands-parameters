\section{The moduli stack of Langlands parameters}
    \begin{convention}
        Henceforth, we shall refer to Langlands parameters as \say{L-parameters} for brevity.
    \end{convention}
    
    \subsection{Tame parameters}
        \subsubsection{The moduli space of tame parameters}
            \begin{convention}
                \item 
                \begin{itemize}
                    \item Throughout, we work with a fixed non-archimedean local field $F$ of residue characteristic $p > 0$ and residue field $\F_q$ (with $q$ being some power of the prime $p$). $W_F$ shall denote the Weil group of $F$, while $I_F$ and $P_F$ shall denote the inertia and wild inertia groups of the local field $F$, respectively.
                    \item In addition, we shall be concerned with Langlands dual groups of reductive groups $G$ over $\Spec E$ (so-called \say{L-groups}), which we shall denote by ${}^LG$. 
                \end{itemize}
            \end{convention}
            \begin{convention}
                Suppose that $K/\Q$ is a number field with ring of integers $K^{\circ}$ and that $\hat{G}$ is a split (Zariski-)connected reductive group over $\Spec K^{\circ}\left[\frac1p\right]$ in which we fix a Borel subgroup $\hat{B} \leq \hat{G}$ along with a split maximal torus $\hat{T} \leq \hat{B}$ therein.
            \end{convention}
        
        \subsubsection{Reduction to the tame case}
        
    \subsection{The moduli stack of L-parameters}
        \subsubsection{Construction}
        
        \subsubsection{Singularities and deformations}
    
        \subsubsection{GIT quotient of the moduli stack of L-parameter by the conjugate action of the Langlands dual group}